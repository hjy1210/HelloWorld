
\documentclass{creport}
\usepackage{metalogo}
\usepackage{xeCJK}
\setCJKmainfont{微軟正黑體}  %textrm
\setCJKsansfont{宋体}	 % textsf
\setCJKmonofont{宋体}  % texttt
\setCJKmathfont{宋体}
\newCJKfontfamily[song]\song{宋体} %Now we can use \song to change chinese font to 宋体
\newCJKfontfamily[kai]\kai{標楷體} %Now we can use \kai to change chinese font to 標楷體
\newCJKfontfamily[hei]\hei{微軟正黑體} %Now we can use \hei to change chinese font to 微軟正黑體
%\xeCJKsetup{AutoFakeBold=true}
%\xeCJKsetup{AutoFakeSlant = true}
%\setCJKmainfont{SimSun}
\begin{document}
\chapter{使用 xeCJK}
\XeLaTeX 擴充 \LaTeX 的功能,變成可以處理以 UTF-8 編碼的萬國碼(Unicode)。xeCJK 讓 \XeLaTeX 處理中文相關的問題更得心應手。
\section{設定中文字型}
要設定中文字型,首先要知道系統有哪些中文字型可供使用。\XeTeX 通常使用 fontconfig 尋找和選用字型,
我們可以用fc-list 命令顯示可用的字體。在 Windows 的命令視窗可以用指令
\begin{verbatim}
  fc-list -f "%{family}\n" :lang=zh > zhfont.txt
\end{verbatim}
將中文的字型資訊存到 zhfont.txt 檔案中。以我的系統為例,部分內容如下
\begin{verbatim}
DFKai-SB,標楷體
Microsoft YaHei,微软雅黑
MingLiU_HKSCS,細明體_HKSCS
Microsoft JhengHei,微軟正黑體,微軟正黑體 Light
Microsoft YaHei,微软雅黑
Microsoft YaHei UI
Microsoft JhengHei UI,Microsoft JhengHei UI Light
Microsoft JhengHei,微軟正黑體
Microsoft JhengHei UI
Microsoft JhengHei UI
Microsoft YaHei UI,Microsoft YaHei UI Light
Microsoft YaHei UI
SimSun,宋体
NSimSun,新宋体
Microsoft YaHei,微软雅黑,Microsoft YaHei Light,微软雅黑 Light
MingLiU,細明體
PMingLiU,新細明體
Microsoft JhengHei,微軟正黑體
\end{verbatim}
有了字體名稱,我們可以在文類的導言區中,用下列指令
\begin{verbatim}
\usepackage{xeCJK}
\setCJKmainfont{微軟正黑體}  %textrm
\setCJKsansfont{宋体}	 % textsf
\setCJKmonofont{宋体}  % texttt
\setCJKmathfont{宋体}
\newCJKfontfamily[song]\song{宋体}
%Now we can use \song to change chinese font to 宋体
\newCJKfontfamily[kai]\kai{標楷體} 
%Now we can use \kai to change chinese font to 標楷體
\newCJKfontfamily[hei]\hei{微軟正黑體} 
%Now we can use \hei to change chinese font to 微軟正黑體
\end{verbatim}
設定在 textrm, textsf, texttt 環境裡分別使用微軟正黑體、宋体、宋体。
而且,
可以用指令 \verb+\song, \kai, \hei+
分別 切換到宋体、標楷體、微軟正黑體。

\section{設定文類(documentclass)}

\LaTeX 本身有article, report, book, beamer 等文類,我們可以略加修改當作適合中文的文類。

以 report 為例,report.cls 在 "MikTex 2.9/tex/latex/base" 資料夾中,複製一份命名creport.cls。
在creport.cls裡面進行下面的修改:
\begin{verbatim}
"\@chapapp\ \thechapter.\ %" 改成 "\@chapapp\ \thechapter 章 \ %" (兩處)
"\@chapapp\space\thechapter." 改成 "\@chapapp\space\thechapter 章" (兩處)
"\partname\nobreakspace\thepart" 改成 "\partname\nobreakspace\thepart\ 部"(一處)
\end{verbatim}
並將 \verb+\contentsname,...,\abstractname+等指令代表的英文字,
如contents,...,abstract等改成目錄,...,摘要等等,如下:
\begin{verbatim}
\newcommand\contentsname{目錄}
\newcommand\listfigurename{圖清單}
\newcommand\listtablename{表清單}
\newcommand\bibname{參考書目}
\newcommand\indexname{索引}
\newcommand\figurename{圖}
\newcommand\tablename{表}
\newcommand\partname{第}
\newcommand\chaptername{第}
\newcommand\appendixname{附錄}
\newcommand\abstractname{摘要}
\end{verbatim}
再將 \verb+\ProvidesClass{report}+ 修改成 \verb+\ProvidesClass{creport}+。

最後,因為 MikTex 利用功能名稱資料庫來加快日常的作業,而我們新加了creport.cls,
必須更新功能名稱資料庫才能正常運作。因此,我們到 MikTex Setting 工具 General 標籤頁裡,
按下 Refresh FNDB 按鈕,完成功能名稱資料庫的更新。

 
\chapter{雜項}
中華民國\textbf{臺灣}\LaTeX \textsf{示例}。

\kai 標楷體 \song 宋体 \hei \textbf{微軟正黑體}
%\CJKfontspec[Scale=0.962216]{細明體} 細明體 % warning : Font 細明體 does not contain script 'CJK', 'Latin' script used instead.
\nocite{*}
\bibliographystyle{plain}
\bibliography{xecjk}

\end{document}