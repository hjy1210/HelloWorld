\chapter{CTeX 文類正體字設定}
在 MikTex2.9/tex/latex/ctex/config 目錄中,ctexops.cfg 與 ctex.cfg 是 \CTeX 的字型設定檔,預設字型是簡體字的微軟雅黑,對使用正體字的人來說,非常不方便。
在不去更動這兩個設定檔的原則下,我們設法使用正體中文字型。
\section{文類選項設定}
\CTeX 的四種文類 ctexrep, ctexart, ctexbook, ctexbeamer,分別與 report, article, book, beamer 相對應。它們的預設字型都是簡體字字型。為了搭配正體字,在文類選項中,要設定 fontset 為 none,以 ctexrep 為例,語法為:
\begin{quote}
$\backslash$documentclass[fontset = none]\{ctexrep\} 
\end{quote}
\section{文類字型組合設定}
在 \CTeX 的文類中,不同的場合使用不同的字型,將下列指令放在導言區(preamble)裡面:
\begin{enumerate}
\item $\backslash$setCJKmainfont\{微軟正黑體\} 
\item $\backslash$setCJKsansfont\{標楷體\},
\item $\backslash$setCJKmonofont\{標楷體\},
\end{enumerate}
來設定與英文normal, san serif, mono 等字型對應的正體中文字型。

\section{切換字型指令設定}
為了讓使用者在文章中可以自由切換字型,在導言區內用:
\begin{itemize}
\item
\begin{enumerate}
\item $\backslash$setCJKfamilyfont\{kai\}\{標楷體\}
\item $\backslash$newcommand\{$\backslash$kai\}\{$\backslash$CJKfamily\{kai\}\}
\end{enumerate}
設定$\backslash$kai 指令,用來切換到正楷體。
\item 
\begin{enumerate}
\item $\backslash$setCJKfamilyfont\{hei\}\{微軟正黑體\}
\item $\backslash$newcommand\{$\backslash$hei\}\{$\backslash$CJKfamily\{hei\}\}
\end{enumerate}
設定$\backslash$hei 指令,用來切換到微軟正黑體。
\end{itemize}
\section{章節圖表名稱設定}
為了讓章節圖表名稱
使用正體字,在導言區內用下面敘述設定:
\begin{verbatim}
\ctexset{
  space=auto,
  autoindent=false,
  section={
    name={第,節},
    number={\arabic{chapter}.\arabic{section}},
  },
  chapter={
    name={第,章},
    number=\chinese{chapter},
  },
  part/name = {第,部分},
  abstractname = 摘要,
  appendixname = 附錄,
  figurename={圖},
  indexname={索引},
  contentsname = {目錄},
  listfigurename = {插圖列表},
  listtablename = {表格列表},
  proofname = 證明 ,
  bibname = {參考文獻}
}

\end{verbatim}
